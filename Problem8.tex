\documentclass[11 pt]{article}
\title{Problem 8}
\usepackage{latexsym}
\usepackage{amssymb}
\usepackage{amsfonts}
\usepackage{amsmath}
\usepackage{amsthm}
\newtheorem{proposition}{Proposition}

\newcommand{\newpar}{\vspace{.15in}\noindent}

\begin{document}

\noindent Jake Irons, MTH 350-02, Problem RF.5, Draft 1

\noindent ironsj@mail.gvsu.edu

\newpar
\begin{proposition}
Let $D_2(\mathbb{R})$ be the set of all 2 x 2 diagonal matrices with real number entries. That is, $D_2(\mathbb{R})= \begin{bmatrix}
    a & 0 \\
    0 & b
  \end{bmatrix}$. Prove that $D_2(\mathbb{R})$ is a subring of $M_2(\mathbb{R})$ (the ring of all 2 x 2 matrices with real number entries).
\end{proposition}
\begin{proof}
To prove $D_2(\mathbb{R})$ is a subring of $M_2(\mathbb{R})$, we must first show $D_2(\mathbb{R})$ is a subset of $M_2(\mathbb{R})$. Let $a, b \in \mathbb{R}$ and $\begin{bmatrix}
    a & 0 \\
    0 & b
  \end{bmatrix}\in D_2(\mathbb{R})$. All of the entries of $\begin{bmatrix}
    a & 0 \\
    0 & b
  \end{bmatrix}$ are real numbers, and therefore, would fit the definition of $M_2(\mathbb{R})$ where it is a 2 x 2 matrix and all of its entries are real numbers. Thus, $\begin{bmatrix}
    a & 0 \\
    0 & b
  \end{bmatrix} \in M_2(\mathbb{R})$, and by definition $D_2(\mathbb{R}) \subseteq M_2(\mathbb{R})$.
  
  \newpar
  Now that we have shown $D_2(\mathbb{R}) \subseteq M_2(\mathbb{R})$, Theorem 9.3 allows us to say if we can show $D_2(\mathbb{R})$ is closed under addition and multiplication, contains $0_R$, and is closed under additive inverses, we can say $D_2(\mathbb{R})$ is a subring of $M_2(\mathbb{R})$.
  
  \newpar
  Let $\begin{bmatrix}
    a & 0 \\
    0 & b
  \end{bmatrix}, \begin{bmatrix}
    c & 0 \\
    0 & d
  \end{bmatrix} \in D_2(\mathbb{R})$. If we are to add the two matrices together, the result is $\begin{bmatrix}
    (a+c) & 0 \\
    0 & (b+d)
  \end{bmatrix}$. Since $a, b, c, d$ are real numbers and the real numbers are closed under addition, $a+c$ and $b+d$ are real numbers. Thus, $\begin{bmatrix}
    (a+c) & 0 \\
    0 & (b+d)
  \end{bmatrix}\in D_2(\mathbb{R})$ and is closed under addition.
  
  \newpar
Now if we are to multiply the two matrices together, the result is $\begin{bmatrix}
    ac & 0 \\
    0 & bd
  \end{bmatrix}$. Since $a, b, c , d$ are real numbers and the real numbers are closed under multiplication, $ac$ and $bd$ are real numbers. Thus, $\begin{bmatrix}
    ac & 0 \\
    0 & bd
  \end{bmatrix}\in D_2(\mathbb{R})$ and is closed under multiplication.
  
  \newpar
  We will once again let $a, b\in \mathbb{R}$ and $\begin{bmatrix}
    a & 0 \\
    0 & b
  \end{bmatrix}\in D_2(\mathbb{R})$. If we add $\begin{bmatrix}
    0 & 0 \\
    0 & 0
  \end{bmatrix}$, which by definition is an element of $D_2(\mathbb{R})$, the result is $\begin{bmatrix}
    a & 0 \\
    0 & b
  \end{bmatrix}$. Therefore, since $\begin{bmatrix}
    a & 0 \\
    0 & b
  \end{bmatrix}$+$\begin{bmatrix}
    0 & 0 \\
    0 & 0
  \end{bmatrix}$=$\begin{bmatrix}
    a & 0 \\
    0 & b
  \end{bmatrix}$, $\begin{bmatrix}
    0 & 0 \\
    0 & 0
  \end{bmatrix}$ is the zero element in $D_2(\mathbb{R})$ by definition.
  
  \newpar
  Finally, we will add $\begin{bmatrix}
    -a & 0 \\
    0 & -b
  \end{bmatrix}$ to $\begin{bmatrix}
    a & 0 \\
    0 & b
  \end{bmatrix}$. We know $\begin{bmatrix}
    -a & 0 \\
    0 & -b
  \end{bmatrix}\in D_2(\mathbb{R})$ since the diagonal entries are real numbers. When we add the two matrices together the result is $\begin{bmatrix}
    0 & 0 \\
    0 & 0
  \end{bmatrix}$, or the zero element in $D_2(\mathbb{R})$. By definition, every element of $D_2(\mathbb{R})$ has an additive inverse, and therefore, is closed under additive inverses. Since we have shown $D_2(\mathbb{R}) \subseteq M_2(\mathbb{R})$ and $D_2(\mathbb{R})$ is closed under addition and multiplication, contains $0_R$, and is closed under additive inverses, Theorem 9.3 tells us $D_2(\mathbb{R})$ is a subring of $M_2(\mathbb{R})$.




\end{proof}
\end{document}