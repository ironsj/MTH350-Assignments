\documentclass[11 pt]{article}
\title{Problem 5}
\usepackage{latexsym}
\usepackage{amssymb}
\usepackage{amsfonts}
\usepackage{amsmath}
\usepackage{amsthm}
\newtheorem{proposition}{Proposition}

\newcommand{\newpar}{\vspace{.15in}\noindent}

\begin{document}

\noindent Jake Irons, MTH 350-02, Problem RF.4, Draft 1

\noindent ironsj@mail.gvsu.edu

\newpar
\begin{proposition}
An element $a$ in a ring $R$ is called idempotent if $a^2 = a$. Prove or disprove: The only idempotent elements in an integral domain are $0_R$ and $1_R$ (that is, the additive and multiplicative identities
of the ring).


\end{proposition}
\begin{proof}
Let $R$ be an integral domain and the element $a \in R$ be idempotent. This means that $a^2=a$. Thus, $a^2-a=0_R$. Using the distributive property $a(a-1_R)=0_R$. By the definition of integral domain, there are no zero divisors. Therefore either $a=0$ or $a-1_R=0$. Furthermore, we can say $a=0_R$ or $a=1_R$. We have then determined the only idempotent elements of $R$ are $0_R$ and $1_R$, completing the proof.

\end{proof}
\end{document}